\documentclass[11pt,a4paper]{article}
\usepackage[utf8]{inputenc}
\usepackage[spanish]{babel}
\usepackage{amsmath}
\usepackage{amsfonts}
\usepackage{amssymb}
\usepackage{mathtools}
\usepackage{cancel}
\usepackage{makeidx}
\usepackage{graphicx}
%\usepackage{times}
%\usepackage{lmodern}
%\usepackage{kpfonts}
\usepackage[left=2cm,right=2cm,top=2cm,bottom=2cm]{geometry}
\usepackage{bm}
\usepackage{cite}
\usepackage{hyperref}
\usepackage{cleveref}
\usepackage{autonum}
\usepackage{comment}
\usepackage{braket}
\usepackage{amsthm}
\usepackage{caption}
\usepackage{subcaption}
\usepackage{tikz}

\usetikzlibrary{babel}

\newcommand{\I}{\text{i}}

\title{Anexo 1}
\author{Erwin Renzo Franco Diaz}
\date{}

\begin{document}
\maketitle



\section{Solución de la ESIT}

\begin{figure}[h]
\centering
\begin{tikzpicture}

\draw[<->] (8.5, 0) node[anchor = west] {$x$}-- (0,0) -- (0,5) node[anchor = south] {$V(x)$};

\draw[blue, line width = 0.5mm] (0,0) -- (0, 4.5);
\draw[blue, line width = 0.5mm] (-0.025,0) -- (2.025,0);
\draw[blue, line width = 0.5mm] (2,0) -- (2,3.025);
\draw[blue, line width = 0.5mm] (2,3) -- (4.525, 3);
\draw[blue, line width = 0.5mm] (4.5, 3) -- (4.5,-0.025);
\draw[blue, line width = 0.5mm] (4.5,0) -- (8,0);

\end{tikzpicture}
\end{figure}

\begin{equation}
	-\frac{1}{2m}\frac{d^2\phi}{dx^2} + V(x)\phi = E\phi = \frac{p^2}{2m} 
\end{equation}

\begin{equation}
	\frac{d^2\phi}{dx^2}= \left(2mV(x) - p^2 \right)\phi
\end{equation}

\subsection{FV}

\begin{equation}
	\frac{d^2\phi_{\text{FV}}}{dx^2} = -p^2\phi_{\text{FV}}
\end{equation}

\begin{equation}
	\phi^{\text{FV}}(x) = A\sin\left(px\right) + B\cos\left(px\right)
\end{equation}
Como se tiene una barrera infinita en $x=0$
\begin{equation}
	\phi_{\text{FV}}(0) = B = 0
\end{equation}

\begin{equation}
	\phi^{\text{FV}}(x) = A\sin\left(px\right)
\end{equation}

\subsection{B}

\begin{equation}
	\frac{d^2\phi}{dx^2}= \left(2mV_0 - p^2 \right)\phi
\end{equation}
Definimos 
\begin{equation}
	\kappa^2 = 2mV_0 - p^2
\end{equation}
con $k > 0$

\begin{equation}
	\phi^{\text{B}}(x) = Ce^{\kappa x} + De^{-\kappa x}
\end{equation}

\subsection{R}

\begin{equation}
	\frac{d^2\phi_{\text{R}}}{dx^2} = -p^2\phi_{\text{R}}
\end{equation}

\begin{equation}
	\phi^{\text{R}}(x) = Ee^{\I px} + Fe^{-\I px}
\end{equation}

\subsection{Continuidad de $\phi(x)$}

\subsubsection{$x =a$}

Continuidad de $\phi(x)$ en $x = a$
\begin{gather}
	\phi^{\text{FV}}(a) = \phi^{\text{B}}(a) \\ \label{eq1}
	A\sin\left(pa\right) = Ce^{\kappa a} + De^{-\kappa a}
\end{gather}
Continuidad de $\phi'(x)$ en $x = a$
\begin{gather}
	\frac{d\phi^{\text{FV}}}{dx}(a) = \frac{d\phi^{\text{B}}}{dx}(a) \\ \label{eq2}	A\cos\left(pa\right) = \frac{\kappa}{p} \left( Ce^{\kappa a} - De^{-\kappa a} \right)
\end{gather}
%Dividiendo \eqref{eq1} entre \eqref{eq2} para eliminar $A$
%\begin{gather}
%	\tan\left(pa\right) = \frac{p}{\kappa}\left( \frac{Ce^{\kappa a} + De^{-\kappa a}}{ Ce^{\kappa a} - De^{-\kappa a} }\right) \\		\left(\kappa\tan\left(pa\right)-p\right)e^{\kappa a}C = -\left(\kappa\tan\left(pa\right) + p\right)e^{-\kappa a}D \\ \label{eq3}
%C = -\left(\frac{\kappa\tan\left(pa\right)-p}{\kappa\tan\left(pa\right) + p}\right) e^{-2\kappa a}D
%\end{gather}
\eqref{eq1} y \eqref{eq2} forman un sistema de ecuaciones lineales a partir del cual se puede obtener $C$ y $D$ en función $A$

\begin{alignat}{3}
&e^{\kappa a}C + &&e^{-\kappa a}D &&= A\sin\left(pa\right) \\
&e^{\kappa a}C - &&e^{-\kappa a}D &&= \frac{p}{\kappa} A\cos\left(pa\right)
\end{alignat}

\begin{align}
C &= \frac{e^{-\kappa a}}{2}A\left(\sin\left(pa\right) + \frac{p}{\kappa} \cos\left(pa\right)\right)\\
D &= \frac{e^{\kappa a}}{2}A\left(\sin\left(pa\right) - \frac{p}{\kappa} \cos\left(pa\right) \right)
\end{align}

\subsubsection{$x = b$}

Continuidad de $\phi(x)$ en $x = b$
\begin{gather}
	\phi^{\text{B}}(b) = \phi^{\text{R}}(b) \\ \label{eq4}
	Ce^{\kappa b} + De^{-\kappa b} = Ee^{\I pb} + Fe^{-\I pb}
\end{gather}
Continuidad de $\phi'(x)$ en $x = b$
\begin{gather}
	\frac{d\phi^{\text{B}}}{dx}(b) = \frac{d\phi^{\text{R}}}{dx}(b) \\ \label{eq5}
	Ce^{\kappa b} - De^{-\kappa b} = \I\frac{p}{\kappa} \left(Ee^{\I pb} - Fe^{-\I pb}\right)
\end{gather}
\eqref{eq4} y \eqref{eq5} forman un sistema de ecuaciones lineales a partir del cual se puede obtener $E$ y $F$ en función de $C$ y $D$ y por lo tanto en función de $A$
\begin{alignat}{3}
    &e^{\I pb}E + &&e^{-\I pb}F &&= Ce^{\kappa b} + De^{-\kappa b} \\
	&e^{\I pb}E - &&e^{-\I pb}F  &&= -\I\frac{\kappa}{p} \left(Ce^{\kappa b} - De^{-\kappa b}\right)
\end{alignat}

\begin{gather}
    E = \frac{e^{-\I bp}}{2}\left(\left(1 - \I \frac{\kappa}{p}\right)Ce^{b\kappa} + \left(1 + \I \frac{\kappa}{p}\right)De^{-b\kappa}\right)\\
    F = \frac{e^{\I bp}}{2}\left(\left(1 + \I \frac{\kappa}{p}\right)Ce^{b\kappa} + \left(1 - \I \frac{\kappa}{p}\right)De^{-b\kappa}\right)
\end{gather}
Como todas las constantes contienen un factor de $A$, esta puede ser absorbida en la constante de normalización por lo que podemos hacerla igual a 1.
\begin{gather}
	C\left(p\right) = \frac{e^{-\kappa a}}{2}\left(\sin\left(pa\right) + \frac{p}{\kappa} \cos\left(pa\right)\right)\\
	D\left(p\right)  = \frac{e^{\kappa a}}{2}\left(\sin\left(pa\right) - \frac{p}{\kappa} \cos\left(pa\right) \right) \\
	E\left(p\right)  = \frac{e^{-\I bp}}{2}\left(\left(1 - \I \frac{\kappa}{p}\right)C(p)e^{b\kappa} + \left(1 + \I \frac{\kappa}{p}\right)D(p)e^{-b\kappa}\right)\\
	F\left(p\right)  = \frac{e^{\I bp}}{2}\left(\left(1 + \I \frac{\kappa}{p}\right)C(p)e^{b\kappa} + \left(1 - \I \frac{\kappa}{p}\right)D(p)e^{-b\kappa}\right)
\end{gather}
%Notamos que $F(p) = E^{*}(p)$

\emph{andreesen define las constantes de manera distinta}

\subsection{Normalización}
Como los autovalores de momento $p$ son continuos
\begin{equation}
	\int_{-\infty}^{+\infty} dx \phi^{*}_{p'}(x) \phi_p(x) = N\left(p\right) \delta(p-p')
\end{equation}

\begin{equation}
\int_{-\infty}^{+\infty} dx \phi^{*}_{p'}(x) \phi_p(x) = \int_0^a dx \left( \phi^{\text{FV}}_{p'}(x) \right)^* \phi_p^{\text{FV}}(x) + \int_a^b dx \left( \phi^{\text{B}}_{p'}(x) \right)^* \phi_p^{\text{B}}(x) + \int_b^{+\infty} dx \left( \phi^{\text{R}}_{p'}(x) \right)^* \phi^{\text{R}}_p(x) 
\end{equation}
Las dos primeras integrales dan un valor finito, por lo que la delta de Dirac solo puede resultar de la última integral en la región R.

\begin{align}
    \int_b^{+\infty} dx \left( \phi^{\text{R}}_{p'}(x) \right)^* \phi^{\text{R}}_p(x)  &= \int_b^{+\infty} dx \left( E \left(p'\right) e^{\I p'x} + F \left(p'\right) e^{-\I p'x} \right)^{*} \left( E \left(p\right) e^{\I px} + F \left(p\right) e^{-\I px} \right) \\
	&= \int_b^{+\infty} dx \left( E^{*} \left(p'\right) e^{-\I p'x} + F^{*} \left(p'\right) e^{\I p'x} \right) \left( E \left(p\right) e^{\I px} + F \left(p\right) e^{-\I px} \right) \\
	&= \int_b^{+\infty} dx \left( E^{*} \left(p'\right) E\left(p\right)e^{\I \left(p - p'\right)x} + F^{*} \left(p'\right) F\left(p\right)e^{-\I \left(p - p'\right)x} + E^{*} \left(p'\right) F\left(p\right)e^{-\I \left(p + p'\right)x} + F^{*} \left(p'\right) E\left(p\right)e^{\I \left(p + p'\right)x} \right)
\end{align}

\subsection{Cálculo de $\Gamma$ usando los Autovalores de Energía}

\end{document}